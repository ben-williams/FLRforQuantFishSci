\begin{frame}\frametitle{What does FLR mean?}

\end{frame}

\begin{frame}\frametitle{Why, oh why?}

\end{frame}

\begin{frame}\frametitle{A brief history of FLR time}

\end{frame}

\begin{frame}\frametitle{FLR development}

\end{frame}

\begin{frame}\frametitle{Mission statement}

The FLR project attempts to develop and provide a platform for
quantitative fisheries science based on the R statistical language. The
guiding principles of FLR are openness, through community involvement
and the open source ethos, flexibility, through a design that does not
constraint the user to a given paradigm, and extendibility, by the
provision of tools that are ready to be personalized and adapted. The
main aim is to generalize the use of good quality, open source, flexible
software in all areas of quantitative fisheries research and management
advice.

\end{frame}

\begin{frame}\frametitle{FLR goal}

Promote the use of Management Strategy Evaluation and similar
simulation-based techniques for the exploration of weakness and
strengths of management options and quantitative advice. The FLR toolset
will allow for relatively simple building, and as efficient as possible
running, of MSE simulations Graphical output tailored to different
audiences

\begin{itemize}[<+->]
\item
  Stock assessment and provision of management advice
\item
  Well tested, robust methods
\item
  Open to detailed inspection
\item
  Data and model validation through simulation
\item
  Risk analysis
\item
  Capacity development \& education
\item
  Promote collaboration and openness in quantitative fisheries science
\item
  Open source
\item
  Community involvement
\item
  R as lingua franca
\item
  Support the development of new models and methods
\item
  Extensible toolset
\item
  Links to other tools (ADMB, BUGS, \ldots{})
\item
  Bring the R community and software to fisheries
\end{itemize}
\end{frame}

\begin{frame}\frametitle{FLR now}

\end{frame}

\begin{frame}\frametitle{Packages in FLR}

\end{frame}

\begin{frame}\frametitle{State of development}

\end{frame}

\begin{frame}\frametitle{More information}

\end{frame}
